\documentclass[letterpaper,11pt]{article}
\usepackage[utf8]{inputenc}
\title{8th Meeting}
\date{8~March 2018}


\begin{document}
\maketitle
\section*{Attendance}
\subsection*{Present}
\begin{list}{}{}
	\item Viet
	\item Jan
	\item Mikael
	\item Markus
	\item Tom
	\item Anne Lise
\end{list}
\subsection*{Valid absence}
\begin{list}{}{}
	\item Magnus
	\item Aleksander
\end{list}

\newpage
\section*{Highlights of the meeting}
\begin{itemize}
	\item Discussion of implementation order
	\item Creation and discussion of IChessPiece, ChessPiece, Piece and Vector2
\end{itemize}

\section*{To do}
\begin{itemize}
	\item Mikael is to finish implementation of ChessPiece
	\item Markus is to finish implementation of Bishop
	\item Tom is to finish implementation of King
	\item Anne Lise is to finish implementation of Knight
	\item Jan is to finish implementation of Queen
	\item Viet is to finish implementation of Pawn
\end{itemize}

\section*{Notes taken under meeting}
Created directory for all programming and code implementation in the git repository.\\
One of the assignment's requirement is to have the application follow the regular rules of chess. Starting with this since it's easy to break down.\\
We will eventually see what's better to be taken care of board or chess piece when we start implementing.\\
Starting with the interface IChessPiece.\\
Created Vector2 class.\\
Created Piece enumeration.\\
Created ChessPiece class.\\
Board handles everything regarding movement of pieces.\\
We figured out that everything that handles movements of the chess pieces needs to have access to the board. Board tells where something is, and also if a move is legal.
\end{document}