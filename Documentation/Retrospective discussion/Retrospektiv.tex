\title{Retrospektiv oppsummering}
\date{\vspace{-5ex}}
\documentclass{article}
\usepackage[utf8]{inputenc}

\begin{document}
\maketitle
Starten av prosjektet var preget av at vi ikke kjente hverandre så godt, så det ble brukt litt tid på 
å bli bedre kjent med hverandre og hverandres erfaringer og ferdigheter. Dette gjenspeilet seg i rollefordelingen, hvor på det første og andre møtet ble roller fordelt ut litt tilfeldig. Vi erfarte i løpet av senere møter at fordeling av oppgaver var mer naturlig etter hva folk hadde jobbet med tidligere. Gruppedynamikken dikterte hvor det var mest hensiktsmessig å plassere de forskjellige rollene som dukket opp.\\\\
Noe som har fungert veldig bra for oss er Discord-chatten vi satte opp for teamet. Dette har gitt oss en måte å kommunisere utenfor møtene, men også på en rask måte. Vi opprettet også en Facebook-gruppe, men for de fleste av oss har det vært greit at team-chatten er separert fra sosiale medier.\\\\
Trello var noe som vi syntes var en god ide, og vi hadde et veldig godt oppsett til å begynne med. I etterkant har vi ikke brukt det så veldig aktivt, nie vi føler er mest på grunn av at vi fortsatt er i planleggingsfasen. Vi håper å få tatt i bruk Trello mer, ettersom det gir en ganske god oversikt.\\\\
Vi klarte å komme frem til et tidspunkt som passet alle, og oppmøte på møtene har jevnt utover vært bra. Noe som var litt til bekymring for noen, og som også ble tatt opp av TA, var dødtid på møter. Det som kunne ha blitt ansett som dødtid på møtene våre har egentlig ikke vært dødtid for oss. På møtene har disse som oftest oppstått mellom diskusjonsøkter av ting som enkeltmedlemmer har tatt opp, og da har tiden mellom diskusjonsøktene blitt brukt effektivt til å notere gjøremål til eller for neste oppmøte, og eventuelt brukt til å gjøre umiddelbare endringer på fellesen. Dette har funket veldig bra for oss.\\\\
Oblig. 2 har hovedsakelig gått ut på å planlegge prosjektet, så selv om alt ser bra ut på papir, har vi egentlig ikke erfaring med å jobbe med prosjekt i team. Vi kommer nok til å erfare utover hva som fungerer og hva som ikke fungerer for oss. Så sånn sett er planen vår å fortsette å bruke Discord og Trello som vår hovedmetode å kommunisere mellom hverandre. Programmeringsmessig bruker vi Java som programmeringsspråk, og IntelliJ/Eclipse som IDE.

\end{document}