\title{Team Plan}
\date{\vspace{-5ex}}
\documentclass{article}
\usepackage[utf8]{inputenc}

\begin{document}
\maketitle

\section*{Roller}
Så langt i prosjektet, har vi fordelt disse rollene som vi tenkte var viktigst å få på plass med det samme. Etterhvert som prosjektet går, fordeler
vi hovedansvaret for enkelte ting i prosjektet etter nødvendighet.\\\\
De fleste av oss hadde mer eller mindre samme erfaring programmeringsmessig, både i forhold til relevante fag vi har hatt og har, men også i forhold til det å programmere i team. Jan var den flittigste brukeren av git på gruppen vår, så han ble utnevnt til git-ansvarlig. Mikael hadde tidligere tatt et programmeringskurs på folkehøyskole hvor de jobbet i grupper på 4 mann, slik at han var mest erfaren på dette området og tok på seg ansvaret med å koordinere kodeoppbyggingen.
\subsection*{Team leder}
Navn: Viet\\
Team leder skal ha generell oversikt over gruppen og prosjektet.

\subsection*{Støttespiller}
Navn: Tom\\
Mer eller mindre left-hand man for team leder. Stepper inn for team leder der hvor det er nødvendig.

\subsection*{Git-ansvarlig}
Navn: Jan\\
Git-ansvarlig holder styr på at git-repositoriet er i orden og løser eventuelle problemer som oppstår med det.

\subsection*{Programmeringsansvarlig}
Navn: Mikael\\
Programmeringsasnvarlig skal ha oversikt over hele koden og dirigerer utformingen av kodestrukturen.

\section*{Git-repositoriet}
For å unngå mest mulig komplikasjoner har vi bestemt at man alltid skal pulle før man prøver å pushe noe til repositoriet. Dette sørger for at lokal kopi av repositoriet alltid er oppdatert i forhold til master branch. For å ha det mest oversiktlig og ryddig som mulig, har vi valgt å dele repositoriet i to deler; dokumentasjon og kode. Dokumentasjonsmappen vil inneholde som navnet tilsier; dokumentasjon av prosjektet, planer, applikasjonsspesifikasjoner, brukerveiledning med m.m. Kodemappen vil inneholde applikasjonskoden.

\section*{Kommunikasjon og møter}
For å oppnå et best mulig samarbeid har vi satt opp et ekstra samlingstidspunkt i uken for å kunne diskutere og jobbe sammen med prosjektet. Vi har også opprettet en Discordchat og Facebook gruppe for å kunne kommunisere med hverandre utenom møtene våre.\\\\
Møtetidspunkt:x
			\begin{list}{-}{}
				\item Tirsdager 1000-1200 v/ VilVite (Gruppetime) 
				\item Torsdager 1000-1200 v/ Høyteknologisenteret
			\end{list}

Vi har også tatt i bruk en nettside som heter Trello. Trello er en nettbasert oppslagstavle for bruk i gruppearbeid. Her setter vi opp arbeidsmål og andre oppgaver som må gjøres. Ved å lage en bruke på siden, kan team medlemmer tagge seg på disse oppslagene for å vise at denne spesifikke delen av prosjektet blir gjort arbeid på. Dette gjør at det ikke blir gjort dobbelt arbeid og vi kan også unngå mergekonflikter.

\section*{Risiko}
Som med all type gruppearbeid er det alltid noen risikoer som kan påvirke sluttresultatet. Den mest fatale risikoen, men minst sannsynlige, er om noen på gruppen trekker seg fra faget. Dette gjør at vi må fordele arbeidsoppgavene til medlemmet som forlot på de gjenværende medlemmene. Hvis dette blir for mye arbeid for de gjenværende, er det mulig at vi må ta vekk funksjonalitet. Dette fører til en annen mindre fatal risiko, men kanskje mer sannsynlig, som er overplanlegging, og da må man droppe noe av de planlagte funksjonene når teamet innser at det blir for mye arbeid.

En annen relevant risiko er datatap. Git skal være et nokså trygt lagringssted, men om noe går galt i forhold til git, så tar hvert team medlem sikkerhetskopi av lokal repositorie av prosjektet jevnlig.

\end{document}