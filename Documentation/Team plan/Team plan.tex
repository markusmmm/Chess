\title{Team Plan}
\date{\vspace{-5ex}}
\documentclass{article}
\usepackage[utf8]{inputenc}

\begin{document}
\maketitle

\section*{Roller}
Så langt i prosjektet, har vi fordelt disse rollene som vi tenkte var viktigst å få på plass med det samme. Etterhvert som prosjektet går, fordeler
vi hovedansvaret for enkelte ting i prosjektet etter nødvendighet.\\
\subsection*{Team leder}
Navn: Viet\\
Team leder skal ha generell oversikt over gruppen og prosjektet.

\subsection*{Git-ansvarlig}
Navn: Jan\\
Git-ansvarlig holder styr på at git-repositoriet er i orden og løser eventuelle problemer som oppstår med det.

\subsection*{Programmeringsansvarlig}
Navn: Mikael\\
Programmeringsasnvarlig skal ha oversikt over hele koden og dirigerer utformingen av kodestrukturen.

\section*{Git-repositoriet}
For å unngå mest mulig komplikasjoner har vi bestemt at man alltid skal pulle før man prøver å pushe noe til repositoriet. Dette sørger for at lokal kopi av repositoriet alltid er oppdatert i forhold til master branch. For å ha det mest oversiktlig og ryddig som mulig, har vi valgt å dele repositoriet i to deler; dokumentasjon og kode. Dokumentasjonsmappen vil inneholde som navnet tilsier; dokumentasjon av prosjektet, planer, applikasjonsspesifikasjoner, brukerveiledning med m.m. Kodemappen vil inneholde applikasjonskoden.

\newpage
\section*{Kommunikasjon og møter}
For å oppnå et best mulig samarbeid har vi satt opp et ekstra samlingstidspunkt i uken for å kunne diskutere og jobbe sammen med prosjektet. Vi har også opprettet en Discordchat og Facebook gruppe for å kunne kommunisere med hverandre utenom møtene våre.\\\\
Møtetidspunkt:
			\begin{list}{-}{}
				\item Tirsdager 1000-1200 v/ VilVite (Gruppetime) 
				\item Torsdager 1000-1200 v/ Høyteknologisenteret
			\end{list}

Vi har også tatt i bruk en nettside som heter Trello. Trello er en nettbasert oppslagstavle for bruk i gruppearbeid. Her setter vi opp arbeidsmål og andre oppgaver som må gjøres. Ved å lage en bruke på siden, kan team medlemmer tagge seg på disse oppslagene for å vise at denne spesifikke delen av prosjektet blir gjort arbeid på. Dette gjør at det ikke blir gjort dobbelt arbeid og vi kan også unngå mergekonflikter.


\end{document}