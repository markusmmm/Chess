\documentclass[letterpaper,11pt]{article}
\usepackage[utf8]{inputenc}
\title{Møte 2}
\date{8.~Februar 2018}


\begin{document}
\maketitle
\section*{Oppmøte}
\subsection*{Tilstede}
\begin{list}{}{}
	\item Viet
	\item Tom
	\item Jan
	\item Mikael
	\item Anne Lise
	\item Magnus
	\item Markus
\end{list}
\subsection*{Gyldig fravær}
\begin{list}{}{}
	\item Aleksander
\end{list}

\newpage
\section*{Høydepunkter}
\begin{itemize}
	\item Ble enig om at kommentering på kode skal foregå på engelsk
	\item Hver enkelt medlem la ut forslag til kodestruktur
	\item Vurdering av bruk av spillmotor
	\item Diskusjon av AI
	\item Diskusjon av funksjonaliteter som folk kom på
	\item Diskusjon av brukermanual
\end{itemize}

\section*{Gjøremål til neste gang}
\begin{itemize}
	\item Bli ferdig med brukermanual
\end{itemize}

\section*{Notater tatt under møte}
Ble enig om å ha applikasjonets språk på engelsk. Kanskje ha et valg om å velge mellom engelsk og norsk?\\
Vi gikk gjennom ideer for hvordan kodeutformingen skulle være, med tanke på hvilke klasser og metoder som skulle være. Mikael hadde laget et klassediagram
som var ganske entydig med det resten av gruppen hadde sett for seg. Vi gjorde småendringer mens vi gikk gjennom den. Vi gikk så gjennom alle user stories for å forsikre  oss om at vi hadde fått med oss det mest essensielle og la til det vi følte vi manglet. Kartlegging av dette har vi på trello. På trello har vi en boks med alle kodetekniske gjøremål som vi har kalt for TODO-LIST. Gjøremålene kan bli markert av hvert enkelt medlem, og markeringen vil indikere at dette teammedlemmet jobber med denne delen av koden slik at det ikke blir duplisering av kode. Vi legger til flere gjøremål som vi arbeider og evaluerer disse neste gang vi møtes.\\
Det ble foreslått om å bruke en spillmotor for å utvikle applikasjonen, men krav om å bruke et annet språk som de fleste av oss ikke har så mye eller noe erfaring med i det hele tatt var kanskje for mye av en ulempe.\\

Vi snakket litt hvordan vi skulle bestemme vanskelighetsgraden på AI, hvor Jan foreslo at vi kunne få AI til å utføre trekk basert på en min-max algoritme for poenguttelling. Dette medfølger at spillerne får poeng for å gjøre trekk i ulike situasjoner og for å ta motstanderbrikker. På lett nivå så vil AI ta trekk nærmere min-enden, middels nivå midt på, og vanskelig på max-enden.\\

Det ble foreslått å ha logg over alle trekk som blir gjort iløpet av et spill, og dette vises på siden.\\
Var også forslag om å ha en oversikt over hvilke brikker en spiller har tatt et sted utenfor brettet (ikon av brikke og så antall rett ved).\\

Utforming av brukermanualen:\\
Bilde av brett og alle de forskjellige brikkene\\
Formelle spillregler\\
Forklaring av alle trekk med bilder (start- og sluttposisjon)\\
Tips og triks/enkle strategier\\
Lage på google docs i første omgang og så få det på latex\\
Link: https://docs.google.com/document/d/16mdqn0RlHeukv9Q8LNaKcsd7D7BqlfJLCfo8me5EhuA/edit?usp=sharing\\

Vi kom frem til at dette vil være "key features" ved applikasjonen vår.\\
- Game difficulties\\
- Highlighting of possible moves\\
- Regret button to go back up to 3 turns\\	
- Move log\\
- Highscore\\
- Reskinning of chess pieces and board\\

Lage tester?\\
Spørre gruppeleder hva er fully dressed use case er, eventuelt lese i boken.\\
Markus skal gjøre research om skrum så vi kan begrunne vårt valg av skrum mer faglig.\\
Tom tar på seg hovedansvaret for å beskrive sjakkreglene.\\

Aleksander nedgradert som gitmaster.\\
Jan fått rolle som gitlord.\\
Viet er teamlord.
\end{document}