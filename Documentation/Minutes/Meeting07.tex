\documentclass[letterpaper,11pt]{article}
\usepackage[utf8]{inputenc}
\title{Meeting 7}
\date{6~March 2018}


\begin{document}
\maketitle
\section*{Attendance}
\subsection*{Present}
\begin{list}{}{}
	\item Viet
	\item Jan
	\item Anne Lise
	\item Mikael
	\item Aleksander
	\item Tom
	\item Magnus
\end{list}
\subsection*{Valid absence}
\begin{list}{}{}
	\item Markus
\end{list}

\newpage
\section*{Highlights of the meeting}
\begin{itemize}
	\item Discussion of changes that needs to be done to last submission
	\item Discussion about possible changes in practical conventions regarding workflow and the git repository
	\item Discussion about user stories and making them into completable tasks during sprints
	\item Discussion about prioritizing tasks
\end{itemize}

\section*{To do}
\begin{itemize}
	\item Read up on assignment properly and ask in team chat if anything is unclear
	\item Read up on git branches, and Jan will prepare a practical tutorial
	\item Bring cards for planning poker
	\item Maybe have a review on test driven development?
\end{itemize}

\section*{Notes taken under meeting}
Things we have to fix:\\
- Use case diagram\\
- Class diagram\\

Review of git branches on thursday. Everybody do research so we can have a practical review and test it out\\
We should have a QA queue section on Trello to ensure proof reading before submission\\
Any changes to the repository? Structure as of now is working fine. We will change it as needed when that time comes.\\
Choose an appropriate tool for communication: Our Discord teamchat has been assigned subchats for specific purposes (General chat, help chat and more)\\
How to assign/take tasks: We will transform user stories and use cases into tasks which we can implement. Plannig poker will then take place in order to estimate the size of the task. If the task is too big, it will be divided into smaller tasks. The tasks should be on a diffuculty and size such that anybody on the team can be assigned it.\\
The tasks should all have a priority level. Tasks with high priority level should be completed before tasks with lower. Trello board will also show priority level.\\
How-tos we know we need:\\
- Git branch\\
Starting on task 2:\\
Going through specifications from last submission and pick out those that meets the requirements of this assignment\\
We will start dividing the requirements into smaller and doable tasks next meeting. The foundations of the application is the board and game manager, so that is what we will start with no matter what.\\
\end{document}