\documentclass[letterpaper,11pt]{article}
\usepackage[utf8]{inputenc}
\title{Møte 1}
\date{6.~Februar 2018}


\begin{document}
\maketitle
\section*{Oppmøte}
\subsection*{Tilstede}
\begin{list}{}{}
	\item Viet
	\item Anne Lise
	\item Magnus
	\item Markus
	\item Tom
	\item Aleksander
	\item Mikael
\end{list}
\subsection*{Gyldig fravær}
\begin{list}{}{}
	\item Jan
\end{list}

\newpage
\section*{Høydepunkter}
\begin{itemize}
	\item Bli kjent med hverandre
	\item Diskusjon av roller
	\item Oppretting av teamchat på Discord og Facebook
	\item Oppretting av ekstra oppmøtetidspunkt
	\item Diskusjon av git og ulike verktøy vi kan ta i bruk
	\item Diskusjon av arbeidsmetode
	\item Diskusjon av kravspesifikasjoner og opprettelse av brukerhistorier
\end{itemize}

\section*{Gjøremål til neste gang}
\begin{itemize}
	\item Forslag til kodestruktur (klassediagram)
	\item Få oversikt over oppgaven
	\item Lære basic sjakk
\end{itemize}

\section*{Notater tatt under møte}
Gruppe ble enig om å bruke java som programmeringsspråk.\\
Gruppe diskuterte hvilke roller det kan være innenfor et programmeringsteam og hva rollens oppgaver kan være.\\
Teamleder: Har oversikt over gruppen og prosjekten.\\
Gitansvarlig: Holder styr på prosjektrepositoriet, tar hånd om branches og merging requests.\\
Programmeringsansvarlig: Dirigerer struktur av koden.\\
Gruppe vil ikke ha hierarkisk gruppe, m.a.o. gruppen ikke stopper opp fordi teamleder ikke gir ut ordre. Gruppe deler oppgaver i mellom seg på møter, men medlemmer skal også kunnne ta hånd om problemer som oppstår i mellom møter.\\\\
Gruppe har opprettet facebookgruppe og discordgruppe for å holde kontakt.\\
Discordlink: https://discord.gg/PTWD3t\\\\
Gruppe snakket litt om forventninger til faget. Dele prosjektet opp i deloppgaver.\\
Enighet om å bruke smidige metoden skrum.\\
2 møter i uken; gruppetime tirsdag 10-12 og torsdag 10-12.\\\\
Commitmeldinger bør være entydige og være klar på hvilke endringer som er blitt gjort.\\
Ikke spam med commits heller.\\
Bruk av git:
\begin{list}{}{}
	\item Delt opp i branches (Kinda kopier)
	\item \begin{list}{}{}
			\item Main er hoved-branch og inneholder til enhver tid fungerende versjon av koden
			\item Sub-branches er kopier av main hvor teamet kan jobbe på med å utvikle ny funksjonalitet uten å ødelegge main
		\end{list}
\end{list}


Planlegging av gjøremål:
\begin{list}{}{}
	\item Muligens sette inn TODO i hovedkoden?
	\item User stories ved å ramse opp alle regler
\end{list}

Utdelte roller
\begin{list}{}{}
	\item Programmeringsansvarlig: Mikael
	\item Gitansvarlig: Aleksander
	\item Sekretær: Viet (meg)
	\item Støttespiller: Tom
	\item User stories (trello): Mikael
\end{list}

Andre roller fyller vi inn iløpet av denne og neste uke etter det vi føler blir naturlig når vi kommer igang og blir bedre kjent med hverandre som gruppe.\\

User stories (Deloppgave 2)
\begin{list}{}{}
	\item Ha en meny med valg mellom single player, multi player, highscores, quit, help, tips \& tricks. Hvis man velger single player, skal det være mulig å velge vanskelighetsgrad.
	\item Velge skins?
	\item Bonde: Kan ikke gå bakover, 2 steg første gang man beveger på den, 1 steg ellers, special move: en passant (drive-by, diagonalt angrep)
	\item Tårn: horisontalt og vertikalt, special move: rokade (kongen og tårn bytte plass, må være på startpos og ingen andre brikker imellom, finnes flere varianter)
	\item Konge: Et steg i alle retninger, kan ikke flytte kongen til en rute som sette han i sjakk matt, hvis kongen ikke kan gå noen retninger så er spillet slutt
	\item Dronning: kan gå i alle retnigner som kongen, men så langt som hun vil
	\item Hest: 2 steg fram, 1 til siden,
	\item Løper: diagonalt
	\item Brikken går tilbake hvis spilleren prøver å gjøre et ugyldig trekk
	\item Beginnermode: viser mulige trekk, angre(bare mot maskiner)
	\item Game mode: Hurtigsjakk?
	\item Score vises bak spillerens navn
	\item Hvit spiller begynner
	\item Mulig å velge farge hvis man spiller på AI, hvis det er 2 ekte spillere, så er player 1 hvit og player 2 svart
	\item AI - hvor mange trekk kan den se frem?
	\item Winner/loser popup
\end{list}
\end{document}