\documentclass[letterpaper,11pt]{article}
\usepackage[utf8]{inputenc}
\title{Meeting 13}
\date{05~April 2018}

\begin{document}
\maketitle
\section*{Attendance}
\subsection*{Present}
\begin{list}{}{}
	\item Viet
	\item Jan
	\item Aleksander
	\item Markus
	\item Tom
	\item Anne Lise
	\item Magnus
	\item Mikael
\end{list}

\newpage
\section*{Highlights of the meeting}
\begin{itemize}
	\item Summary of last assignment
	\item Planning poker
	\item Identifying missed milestones from last assignment that needs to be completed ASAP
	\item Starting to chop up ranking and database requirement into smaller tasks
\end{itemize}

\section*{To do}
\begin{itemize}
	\item Mikael fixing concurrentModificationException
	\item Everyone write general description of methods they have written (commenting/JavaDoc)
\end{itemize}

\section*{Notes taken under meeting}
We are going to use planning poker to identify size of requirements\\
We will divide the requirements into smaller tasks accordingly\\
Master branch will be our hand in branch\\
wip branch will be our release branch\\
Each member make a new branch from wip whenever working on a new feature\\\\
Planning poker results:\\
- Implement AI: 9\\
- Sound effects and/or animations: 2\\
- Ranking (database): 7\\
- Offer best move advice for novices (after AI): 9\\ 
- Start chess game in random state: 5\\
- Implement tests: 7\\
- Double login (local multiplayer): 3
- Javadoc: 1\\\\
Things we have to get done from last assignmet:\\
- Castling\\
- Pawn promotion\\
Also need to fix getUsablePieces, it causes concurrentModificationeEception\\
Fix easy and medium AI\\\\
Cutting up ranking:\\
Starting rank score is 1000\\
Computing of ranking and database setup separate\\
Computing of ELO-rating: https://metinmediamath.wordpress.com/2013/11/27/how-to-calculate-the-elo-rating-including-example/\\
\end{document}
