\documentclass[letterpaper,11pt]{article}
\usepackage[utf8]{inputenc}
\title{Møte 3}
\date{13.~Februar 2018}
	

\begin{document}
\maketitle
\section*{Oppmøte}
\subsection*{Tilstede}
\begin{list}{}{}
	\item Viet
	\item Jan
	\item Mikael
	\item Aleksander
	\item Markus
	\item Tom
	\item Magnus
	\item Anne Lise
\end{list}

\newpage
\section*{Høydepunkter}
\begin{itemize}
	\item Formel gjennomgang av delopgpave 4
	\item Så gjennom Markus sitt skriv av deloppgave 3
	\item Så gjennom brukermanual av Tom
	\item Diskusjon av use case diagram og fully dressed use cases
	\item Diskusjon av highscore
\end{itemize}

\section*{Gjøremål til neste gang}
\begin{itemize}
	\item Forberede dokumentasjon vi har gjort oss ferdig med for presentasjon
	\item Markus skriver oppsummering av deloppgave 4
	\item Viet begynner å skrive på use case diagram og fully dressed use cases
\end{itemize}

\section*{Notater tatt under møte}
Formell gjennomgang av deloppgave 4:\\
- Oppmøte: tirsdag og torsdag\\
- Kommunikasjon via discord (fb om nødvendig)\\
- Bruker git for filhåndtering\\
- Fordeler oppgaver etter erfaring\\\\
Markus skrev en fin oppsummering for deloppgave 3.\\
Tom skrev fin oppsummering av sjakkreglene - Ferdig med brukermanual, erstatte placeholders med screenshots fra egen applikasjon når den er klar.\\\\
Fikk oppklaring på use case diagram og fully dressed use case\\
- Use case diagram viser forhold mellom en bruker og et system. Viser alle mulige interaksjoner mellom bruker og system.\\
- Fully dressed use case er nøye beskrivelse av hver enkelt interaksjon mellom bruker og system. Er nærmest pseudokdoe.\\\\
Mikael kom på en mulig løsning på highscore. Testmodell ute på discord\\
Spiller(e) må taste inn navn enten før eller etter et spill er slutt\\
Separat highscore for singleplayer og multiplayer. På singleplayer er det forskjellige score multipliers for å reflektere vanskelighetsgrad\\
Rating system?\\\\
Forskjell på team-plan og prosjekt-plan:\\
- Team-plan - rollefordeling, kommunikasjon\\
- Prosjekt-plan - planlegging av gjennomførelse\\\\
Muligens ha git workshop med teamet så det blir enklere å bruke?
\end{document}