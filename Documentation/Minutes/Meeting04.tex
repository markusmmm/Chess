\documentclass[letterpaper,11pt]{article}
\usepackage[utf8]{inputenc}
\title{Møte 4}
\date{15.~Februar 2018}


\begin{document}
\maketitle
\section*{Oppmøte}
\subsection*{Tilstede}
\begin{list}{}{}
	\item Viet
	\item Tom
	\item Markus
	\item Mikael
	\item Aleksander
	\item Jan
	\item Anne Lise
	\item Magnus
\end{list}

\newpage
\section*{Høydepunkter}
\begin{itemize}
	\item Retrospektiv diskusjon av prosjektet så langt
	\item Formell gjennomgang av team plan
\end{itemize}

\section*{Gjøremål til neste gang}
\begin{itemize}
	\item Bli ferdig med gjenværende dokumentasjon (deloppgave 4 og 5, fully dressed use case)
	\item Fortsette med presentasjonsforberedning
	\item Begynne å få dokumentasjon på LaTeX
\end{itemize}

\section*{Notater tatt under møte}
Retrospektiv oppsummering\\
Hva gikk bra?\\
Bra oppmøte.\\
Ingen arbeid mistet.\\\\
Hva gikk ikke som forventet?\\
Vi er bare i planleggingsfasen, og det har hovedsakelig gått ut på å bli kjent med hverandre, få oversikt over oppgaven som går ut på å planlegge prosjektet, og planlagt prosjektet. Siden vi stort sett bare har planlagt, er det ikke så mye som kunne ha gått galt.\\\\
Hva gikk dårlig?\\
Vanskelig med arbeidsfordeling første oppmøte. Ingen kjente hverandre så det ble mest snakking til å begynne med.\\
Ikke altfor godt kjent med git, merge conflict oppstod som vi ikke helt visste hvordan vi skulle løse\\\\
Team Plan\\
Roller:\\
Så langt i prosjektet, har vi fordelt disse rollene som vi tenkte var viktigst å få på plass med det samme. Etterhvert som prosjektet går, så fordeler vi hovedansvaret for enkelte ting i prosjektet etter nødvendighet.\\
Teamlord og latexlord: Viet\\
Støttespiller: Tom\\
Git-repeleder: Jan\\
Pogrammeringsansvarlig: Mikael\\
Grafikk: Magnus\\\\
Oppmøte: tirsdager og torsdager\\
Kommunikasjonsmetode: Discord (Eventuelt facebook)\\\\
Hva kan gå gale og hvordan har vi tenkt å reagere på dette?\\
Preventiv: lagre lokalt arbeid mer enn et sted, separat. Push ofte.\\
Hvis noen dropper ut: Flere møter, fordele rollen er vacant på de gjenværende. Hvis det er for mye arbeid, droppe funksjon.\\\\
Commit-konvensjon:\\
- Kort og presist. Skal kunne se med en gang hva slags arbeid som er gjort.\\\\
Kommentarer i kode:\\
- Et språk\\
- I første omgang kommentere selvom det er selvfølge hva det er\\\\
Hvis det er problemer med spesifikke kodesnutter, ta det på gruppetimer og diskutere på fellesen.\\\\
Git-repostruktur: Dokumenteringsmappe og programmeringsmappe, alltid pull før start med arbeid. Pull og push ofte.\\\\
På trello: Alle lager kort om hva de tenker å jobbe med før de jobber.
\end{document}