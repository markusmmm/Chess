\documentclass{article}
\begin{document}
\title{Prosess- og prosjektplan}
\author{Gruppe 3}

\maketitle


\raggedright

\textbf{PROSESS -deloppgave 3:} To execute and complete a project of this size, is something that this group has very little experience with. Since we don't know what to expect in the coming months, it is in our best interest to use an agile programming method. It is difficult for us to plan in a traditional waterfall style when we have never had the experience in working with big teams. This project will have aggressive deadlines and high complexity, which requires us to have continuous planning and continuous evolution of both the project and the software. Agile methods are perfect for this. \newline
\newline
The agile method we have decided to choose is scrum. Scrum teams does not typically have a team leader who decides how a problem is solved or delegates tasks, rather they have a team who equally decides how to turn product backlog items into ready products. This suits our group well, since we have decided to make all our decisions in plenary. \newline

Every team who uses scrum needs to have a to-do list, also called a product backlog. For this we use the web-based project management application, Trello. This allows us to keep track of which tasks that needs to be done, who's doing what, and which tasks that has been completed. Tasks from the product backlog will be completed in so called sprints. A scrum sprint is a time-boxed period where specific work is done. Sprints are usually 2-4 works long, which is typically the time interval of an obligatory assignment. This is another reason for why scrum would be a good fit for this project. \newline

Another important ceremony in scrum, is the daily stand-up. This is a short meeting (about 15 minutes) where the team members cover their progress since the last meeting and plan their work for the next meeting. In our team, we won't be able to have daily meetings, because we all have different and complex schedules. A solution to this is Discord. Discord is a program where we can have daily chats, much like the daily stand-ups in scrum. In addition, we have agreed to meet every Thursday, 10:15. This way we can have a "show-and-tell". Show each other what has been done, and discuss what needs to be done.
\newline

\textbf{PROJECT- deloppgave 4(diskutere tirsdag 13. Februar) :}


\begin{thebibliography}{9}

\bibitem{whatisscrum}
 
  \textit{https://www.versionone.com/agile-101/what-is-scrum/},
  What is Scrum?
  2018

\end{thebibliography}


\end{document}